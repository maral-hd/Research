
\subsection{Data and Analysis}
\label{sec:data1}

In this section, we present the results of our experiment
and data analyses.
We discuss further implications of the data
and results for the experiment in Section~\ref{sec:discussion}.

\begin{table}[ht]
\caption{\small Results for Path Generation}
{\small
\begin{center}
\begin{tabular}{|l|c||c|c|c|c|}
\hline
Application & Version Pair & Number of Paths & Number of Paths & Inputs & Reduction \\
 &   & for Entire Application & Generated & Required for & Rate\\
 &   & (path-entire) & by PARTE & path-parte & (path-parte over \\
 &   &  & (path-parte) &  & path-entire) \\
\hline\hline
{\em FAQForge} &  1.3.0 \& 1.3.1 & 73 & 5 & 25 & 93\% \\ \hline
&   1.3.1 \& 1.3.2 & 73 & 19 & 210 & 74\% \\
\hline \hline
{\em osCommerce} &2.2MS1 \& 2.2MS2 & 2403 & 1719 &4392 & 28\%\\ \hline
&2.2MS2 \& 2.2MS2-060817 & 2409 & 58 & 813 & 98\%   \\ \hline
\hline
{\em phpScheduleIt} &  1.0.0 \& 1.1.0 & 481& 338 & 2389 & 30\% \\
\hline
&   1.1.0 \& 1.2.0 & 518& 314 & 3877 & 39\% \\
\hline
&   1.2.0 \& 1.2.12 & 529& 154 & 2339 & 71\% \\
\hline \hline
{\em Mambo} &4.5.5 \& 4.5.6 & 1357 & 65 &490 & 95\% \\
\hline
&4.5.6 \& 4.6.1 & 1388& 1388& 4446 & 0\%  \\
\hline
&4.6.1 \& 4.6.2 & 1416 & 236 & 2075 & 83\%  \\
\hline
&4.6.2 \& 4.6.3 & 1409 & 92 & 1236 & 93\%  \\
\hline
&4.6.3 \& 4.6.4 & 1444 & 114 & 1567 & 92\%  \\
\hline
&4.6.4 \& 4.6.5 & 1444 & 20 & 324 & 99\%  \\
\hline \hline
{\em Mantis} &1.1.6 \& 1.1.7 & 3482 & 221 &1524 & 94\% \\
\hline
&1.1.7 \& 1.1.8 & 3482 & 185 & 1238 & 95\% \\
\hline
&1.1.8 \& 1.2.0 & 4345 & 2802 & 20425 & 36\% \\
\hline
&1.2.0 \& 1.2.1& 4373 & 166 & 1772 & 96\% \\
\hline
&1.2.1 \& 1.2.2 & 4389 & 106 & 1042 & 98\% \\
\hline
&1.2.2 \& 1.2.3 & 4403 & 103 & 890 & 98\% \\
\hline
&1.2.3 \& 1.2.4 & 4419& 199 & 2643 & 95\% \\
\hline

\end{tabular}
\end{center}
\label{tab:results}
}
\end{table}

Table \ref{tab:results}\footnote{There were
a few bugs in the implementation of the path generation algorithm
in our previous work which were fixed before collecting data
for this study. Due to this reason, the data values presented
in this paper do not match those published in our previous
work~\cite{marback12}.} summarizes the data gathered from
our experiment running on {\em FAQForge}, {\em osCommerce},
{\em phpScheduleIt}, {\em Mambo}, and {\em Mantis}.
The table lists, for each application, ``Version Pair''
(two versions of the applications analyzed), ``Number of Paths
for Entire Application (path-entire)'' (the total number of 
linearly independent paths for the latter version), ``Number 
of Paths Generated by PARTE (path-parte)'' (the number of 
linearly independent paths generated using program slices 
for the latter version), ``Inputs Required for path-parte'' 
(the number of input values required for executable test paths),
and ``Reduction Rate'' (test path reduction rate for our approach
(path-parte) over the control technique (path-entire)). 
The number of inputs presented in the table is the summation for
the number of inputs required for each path, thus redundant inputs 
may exist. 

When we consider the entire application for test path generation,
for the latest version of {\em FAQForge}, 73 test paths were 
generated; in the case of {\em osCommerce}, 2409 test paths 
were generated; for {\em phpScheduleIt}, there were 529 test paths; 
{\em Mambo} ahd 1444 test paths; and for {\em Mantis}, 4419 test 
paths were generated. As expected, the number of paths is proportional 
to the size of each application. When we generated test paths by 
applying our approach (path-parte), on average, the test path 
reduction rates over path-entire for the five applications were 
84\%, 63\%, 46\%, 77\%, and 87\%, respectively (in the 
order the applications appeared in the table). 
The following subsections present data analysis for each application.
 
\subsubsection{Results for FAQForge}

For {\em FAQForge}, our test path generator created 5 and 19 
paths for the two pairs of versions, respectively, while the control
technique produced 73 paths for both versions. 
For {\em FAQForge}, the size of the application is relatively small 
compared to other applications, and the changes between versions 
are also very small; therefore, the number of test paths required 
for the new version is relatively small.

Upon manual inspection of the source for the first pair (versions 1.3.0 
and 1.3.1), we discovered that only one of the files in version 1.3.1 
had been changed from version 1.3.0. The modified file was 
a library file that contained functions included by the main index file. 
During path generation, we did not analyze any of the library files 
directly. Instead, the path generator analyzed files that the user 
would execute directly. If {\em FAQForge} files
were not executed directly, they were located in the ``lib'' directory.
Changes made to the library file were propagated to the index file 
during the file preparation phase of preprocessing. 

The second pair for {\em FAQForge} (versions 1.3.1 and 1.3.2) yielded
19 paths using the path generator. Upon manual inspection of the source 
code, unlike the first version pair, it had many changes in the
source files; we found that 12 files in version 1.3.2 had changed from 
version 1.3.1.  Among these 12 files, only three were analyzed directly. 
The rest of files were, again, library files (in the ``lib'' directory)
invoked using the \textit{include} and \textit{require} functions.

As shown in Table~\ref{tab:results}, our technique produced a 
relatively small number of test paths compared to the control technique. 
Our technique reduced the number of test paths by 93\% and 74\% 
for versions 1.3.1 and 1.3.2, respectively. 
The number of inputs required for path-parte for {\em FAQForge}
was small compared to other applications; for each version pair,
the number of inputs was 25 and 210, respectively.

\vspace*{-2pt}
\subsubsection{Results for osCommerce}
\vspace*{-2pt}

For {\em osCommerce}, the test path generator created 1719 and 58
paths for the two pairs of versions, respectively. 
Upon manual inspection of the source for the first pair (versions 2.2MS1 
and 2.2MS2), we found that 279 of the 506 files had changed. 
The modified files were in every module of the application, and the files 
with the largest differences were the library files that were included in 
the executable files. Again, during path generation, we did not analyze 
any library files (files in {\em osCommerce}'s ``include'' directory)
directly. Instead, the path generator only analyzed files that the user 
would execute directly. 

The second pair for {\em osCommerce} (versions 2.2MS2 and 2.2MS2-060817) 
yielded 58 paths using the path generator. Upon manual inspection of 
the source code,  we found that 105 of the 502 files had changed. 
The number of changed files was smaller than the first version pair.
Considering the number of files that were changed for the second version
pair (105 of 502 files), the number of paths generated was relatively 
small (58). The reason was that there were numerous changes in statements 
that contained no variables, and therefore, there were no data 
dependencies. 

As shown in Table~\ref{tab:results}, our technique produced a relatively 
small number of test paths compared to the control technique. 
Our technique reduced the number of test paths by 28\% and 98\% 
for versions 2.2MS2 and 2.2MS2-060817, respectively. 
The input numbers were proportional to the test path number, so
similar to the test path generation results, the number of inputs
required for the version pairs was quite different: 4392 for the first
version pair and 813 for the second version pair.
However, when we consider the ratio between input numbers and test path 
numbers, the first version required a relatively small number of inputs
per path. This result can be attributed to many changes in a simple output 
statement that has no data dependencies. A common example in PHP would 
be to \textit{echo} or \textit{print} static HTML statements.
If the static text changed, we marked the statement as a difference.

\subsubsection{Results for phpScheduleIt}

In the case of {\em phpScheduleIt}, overall,
the test path reduction rates achieved by applying our approach
were low compared to those of other object programs.
In particular, the first two version pairs produced 30\% and 39\%
reduction rates, respectively.
Upon manual inspection of the source code for the first two version
pairs, we found that they involved large functionality 
changes and bug fixes. For the first version pair, among 90 files, 
71 files were changed between the pair. For the second version pair, 
among 143 files, 105 files were changed. 
Due to these extensive changes among files, many code
components needed to be tested, and this was the major reason
for the low test path reduction rates.

For the last version pair, the reduction rate was 71\%, which 
was better than the previous version pairs, but it was slightly 
low compared to other applications.
Our manual inspection found that, among 178 files, 139 files were changed.
While the number of modified files was similar to the second version
pair, the files that contained a large number of code involved minor changes,
so the affected code area was not large; as a consequence, the test path 
reduction rate was higher than other version pairs.
The number of inputs required for path-parte for each version pair
was 2389, 3877, and 2339, respectively.

\subsubsection{Results for Mambo}

For {\em Mambo}, the test path reduction rates for all version pairs
but one were high, ranging from 0\% to 99\%. In particular,
for the last version pair, the reduction rate achieved by applying
our approach was 99\%. This fact indicated that less than 1\%
of test cases are actually required for the modified web application,
a huge savings considering the application. (Note
that the size of the last version of {\em Mambo} is over 133 KLOC.)

Our manual inspection showed that a few minor changes were
introduced between those versions (e.g., 4.6.4 and 4.6.5). Only 14 files
were changed among the 663 files for the version pair. As a result, only
20 test paths were needed for the new version.
Version pair 4.5.5 and 4.5.6 also produced a high reduction rate.
We found that 319 files were changed among the 703 files for this version 
pair, so a high reduction rate was unlikely. Further inspection of 
the source code revealed that most changes were not in the code 
part but in the top comment section for the disclaimer. Thus, these changes
did not affect the number of test paths required for version 4.5.6. 

Unlike other version pairs, for 4.5.6 and 4.6.1, we did not gain any
benefits by applying our technique. By looking at their version number 
changes, we can infer that there were major changes between these 
two versions. In fact, version 4.5.6 was released on 22 Jan 2008, and 
the next version (4.6.1) was released after 10 months (04 Oct 2008), which
indicated that there was a major change between these two versions.
Upon manual inspection of two versions, we also found that a large 
area of code had been changed. Among the 719 files, 317 files were changed, 
and most of the changes were extensive in nature. Most of the functions 
were changed as part of refactoring and the feature update.
These changes required to generate test paths for the entire application.
As a result, a large number of inputs (4446) was also required for those 
test paths. 

\subsubsection{Results for Mantis}

Similar to the {\em Mambo} results, the test path reduction rates
for {\em Mantis} were very high with one exception (the version pair
1.1.8 and 1.2.0). The rates ranged from 36\% to 98\%, and except 
for the third version pair, our approach achieved over 90\% of 
the reduction rate. For these version pairs that produced over 90\% 
of the reduction rate, we found minor functionality changes and bug 
fixes in the source code. There were no big changes in any module of 
the source code. We also inspected the version pair 1.1.7 and 1.1.8
which produced 95\% reduction rate. There were only four minor bug 
fixes and one translational update.  Version 1.1.8 was intended for 
wrapping up minor things before developing the next major release, 1.2.0.

For only one version pair (1.1.8 and 1.2.0), the reduction rate (36\%) 
was low. By inspecting the release date and release notes 
for this version pair, we found that it was a major release
which included large functionality improvement, a large number of bug fixes, 
and architectural restructuring. Manual inspection of the source code also 
reaffirmed the extensive changes. Among the 496 files, 260 files were changed.
For these reasons, almost all components in the modified version needed 
to be tested.

