\vspace*{4pt}
\section{Empirical Study}
\label{sec:study}

As stated in Section \ref{sec:introduction}, the goal of
this study is to assess our regression testing approach in terms 
of the number of new test cases generated by our approach.
In particular, we address the following research:

\begin{itemize}
\item[RQ:] Is the PARTE approach effective in reducing the
number of new test cases necessary for regression testing of
web applications? 
\end{itemize}

\noindent
To investigate our research question, we performed
an empirical study.
The following subsections present our objects of analysis, 
study setup, threats to validity, and data analysis.

\subsection{Objects of Analysis}
\label{sec:objects}

In this study, we used five open source web applications
written in PHP as objects of analysis; the applications
were obtained from SourceForge.

{\em osCommerce}~\cite{oscommerce} (open source Commerce) is 
a web based store-management and shopping cart application.
{\em FAQForge}~\cite{faqforge} is a web application used to 
create FAQ (frequently asked questions) documents, manuals,
and HOWTOs. We used three versions for these two applications. 

{\em phpScheduleIt}~\cite{phpscheduleit},
is a web application that attempts to solve the problem of
scheduling and managing resource utilization. It provides
a permission-based calendar that allows users to self-register
and to reserve resources and tools to manage those reservations.
Some typical applications are a conference room, equipment, or work
shift scheduling. We used four versions of this application. 

{\em Mambo}~\cite{mambo} is a content management
system that can be used for everything from simple websites to complex
corporate applications. It is used worldwide to power
government portals; corporate intranets and extranets; ecommerce sites;
and nonprofit outreach, school, church, and community sites. We used
seven versions of {\em Mambo}.

{\em Mantis}~\cite{mantis} is a web-based bugtracking system. Mantis 
supports multiple
DBMS, such as  the MySQL, MS SQL, and PostgreSQL databases, and works
on multiple platforms. It provides various bugtracking functionalities,
including changelog support, source control integration, and time tracking.
Due to its complicated functionalities, Mantis is the largest one among
the applications with which we experimented. (The latest version is over 200KLOC.)
For this object program, we used eight versions.

They are real, non-trivial web applications that 
have been utilized by a large number of users.
Table \ref{tab:objects} lists, for each of our objects,
the associated ``Version,'' ``Lines of Code,''
and ``No. of Files.''
The lines of code counted in this table include both PHP code and
HTML markups.

\begin{table}[ht]
\caption{Objects of Analysis}
{\small
\begin{center}
\begin{tabular}{|c|c|c|c||c|c|c|c|}
\hline
Application & Version & Lines of Code & No. of Files
&Application & Version & Lines of Code & No. of Files \\
\hline\hline
{\em FAQForge} & 1.3.0 & 1806 & 20 & 
{\em phpScheduleIt} & 1.0.0 & 35045 & 90  \\ \hline
& 1.3.1 & 1837 & 20 & 
& 1.1.0 & 59753 & 143 \\ \hline
& 1.3.2 & 1671 & 18 & 
& 1.2.0 & 63138 & 178 \\ \hline
{\em osCommerce} & 2.2MS1& 53510 & 302 &
& 1.2.12 & 72396 & 192 \\ \hline
& 2.2MS2& 68330 & 506 & 
{\em Mantis} & 1.1.6 & 139124 & 496 \\ \hline
& 2.2MS2-060817& 78892 &  502 & 
& 1.1.7 & 139196 & 496 \\ \hline
{\em Mambo} & 4.5.5 & 149868 & 703 &
& 1.1.8 & 139194 & 496 \\ \hline
& 4.5.6 & 150967 & 719 &
& 1.2.0 & 206150 & 748 \\ \hline
& 4.6.1 & 127309 & 771 &
& 1.2.1 & 206492 & 747 \\ \hline
& 4.6.2 & 129235 & 659 &
& 1.2.2 & 207123 & 746 \\ \hline
& 4.6.3 & 130716 & 654 &
& 1.2.3 & 209104 & 753 \\ \hline
& 4.6.4 & 133420 & 663 &
& 1.2.4 & 209345 & 753 \\ \hline
& 4.6.5 & 133475 & 663 & \multicolumn{4}{|c|}{} \\\hline
\end{tabular}
\end{center}
\label{tab:objects}
}
\end{table}

\subsection{Variables and Measures}
\label{sec:measures}

\subsubsection{Independent Variables}
\label{sec:independent}

Our empirical study manipulated one independent
variable, test generation technique.
We considered one control technique and one heuristic technique.

The control technique generates all possible linearly independent
paths without using any particular regression test case generation
heuristics. This technique serves as an experimental control.  
The heuristic technique generates regression test cases using
the program slices explained in earlier sections. 

\subsubsection{Dependent Variable and Measures}
\label{sec:dependent}

Our dependent variable is the number of test paths generated
by the techniques.

\subsection{Experiment Setup}
\label{sec:setup}

We performed our study using a virtual machine on multiple hosts. 
Because we used several virtual machine hosts and their performance 
capabilities were different, we did not collect data associated with 
time. The operating system for the virtual machine was 
Ubuntu Linux version 10.10. The server ran Apache as its HTTP 
server and MySQL as its database backend.
We used PHP version 5.2.13. 

All but one of the modules in the tools of PARTE were written in Java, 
and the Oracle/Sun JRE and JDK version 6 were used as the development 
and execution platforms. The test execution engine was written in Java. 
PHC version 0.2.0.3 was used to parse the PHP files. Perl and Bash 
scripts were used to control the modules and to pass data.

\subsection{Threats to Validity}
\label{sec:validity}
\vspace*{-3pt}

In this section, we describe the internal and external
threats to the validity of our study. We also describe 
the approaches we used to limit the effects of these threats.

{\em Internal Validity:}
The outcome of our study could be affected by the choice of 
program analysis. We applied static analysis on a dynamic programming 
language, PHP. To do so, we had to change many statements in the 
program during the file preparation phase, removing and fixing 
many dynamic environment variables. However, we carefully examined 
our process to minimize the adversary effect that might be introduced
into the converted files.  

{\em External Validity.}
We used open source web applications for our study, so these programs 
are not representative of the applications used in practice, and thus
we cannot generalize our results. However, we tried to reduce this 
threat by using five non-trivial sized web applications with multiple 
versions that have been utilized by many users. 


