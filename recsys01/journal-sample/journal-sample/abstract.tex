\begin{abstract}

Companies that provide
web applications often encounter various security
attacks and frequent feature update demands from users,
and when these needs arise, companies need to fix security
problems or upgrade the application with new features.
These fixes often involve small patches
or revisions, but still, testers need to perform regression
testing on their products to ensure that the changes have not 
introduced new faults. Performing regression testing on the entire 
product, however, can be very expensive, and it is not a viable  
solution for the companies who need a short turnaround time to  
release patches. One solution is focusing only on the areas of 
code that have been changed and performing regression testing 
on them. By doing this, companies can provide quick patches more 
dependably whenever they encounter security breaches. In this paper, 
we propose a new regression testing approach that identifies 
the affected areas by code changes using impact analysis and 
generates new test cases for the impacted areas by changes using 
program slices. To facilitate our approach, we implemented 
a PHP Analysis and Regression Testing Engine (PARTE) and performed 
a controlled experiment using five open source web applications
with multiple versions.
The results show that our approach is effective in reducing 
the cost of regression testing for a frequently patched web 
application, and they also expose ways in which that 
effectiveness can vary with application characteristics
and versioning frequencies.  

\end{abstract}

