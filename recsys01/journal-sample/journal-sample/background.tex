\section{Related Work}
\label{sec:related-work}

To date, many regression testing techniques have been proposed,
and most of them have focused on reusing the existing test cases 
for regression test selection (e.g., \cite{rothermel97apr, yoo07})
and test case prioritization (e.g., \cite{elbaum02feb, walcott06}).
Existing test cases, however, are often insufficient to retest code 
or system behaviors that are affected by code changes.

To address this problem, recently, 
researchers have started working on test suite augmentation
techniques which create new test cases for areas that have been
affected by changes~\cite{apiwattanapong06, santelices11, xu10, 
xu11, taneja11, chen07}. Apiwattanapong et al.~\cite{apiwattanapong06} 
and Santelices et al.~\cite{santelices11} present an approach to 
identify areas affected by code changes using program dependence 
analysis and symbolic execution, and provide test requirements 
for changed software. Xu et al.~\cite{xu10, xu11} present an approach 
to generate test cases by utilizing the existing test cases and 
adapting a concolic test case generation technique. 
Taneja et al. \cite{taneja11} propose an efficient test
generation technique that uses dynamic symbolic execution, eXpress, 
by pruning irrelevant paths. These approaches focus on desktop
applications such as C or Java, but our approach applies regression
testing to web applications, creating different challenges
(e.g., dynamic programming languages deployed using multi-tier architecture).

Creating executable test cases automatically has been one of 
the challenging tasks for the software testing community~\cite{demillo91}, 
and recent research in this area has proposed a promising approach which 
involves concolic testing (a combination of concrete and symbolic 
execution)~\cite{sen05}. Since then, other researchers have extended 
the concolic technique to handle PHP code~\cite{wassermann08jul, artzi10}.
Wassermann et al.~\cite{ wassermann08jul} and Artzi et al.~\cite{artzi10} 
have also utilized a concolic approach to generate test cases for PHP 
web applications. Other test case generation techniques for web applications 
use crawlers and spiders to identify  and test web application interfaces. 
Ricca and Tonella~\cite{ricca01} use a crawler to discover the link structure
of web applications and to produce test case specifications from this
structure. Deng et al.~\cite{deng04} use static analysis
to extract information related to URLs and their parameters for
web applications, and to generate executable test cases using the
collected information. Halfond et al.~\cite{halfond09} present an
approach that uses symbolic execution to identify precise interfaces
for web applications. 

Other researchers have investigated test case problems considering 
models of the specifications of the system rather than source 
code~\cite{chen07, korel02, fourneret11, naslavsky10, samuel09}. 
Chen et al.~\cite{chen07} present a model-based approach that generates 
a regression test suite using dependence analysis of the modified elements 
in the model. Similarly, Fourneret et al.~\cite{fourneret11} generate 
necessary test cases for the modified portion of the model (UML) using 
requirements informations and impact analysis. Samuel and 
Mall~\cite{samuel09} generate test cases from UML activity diagrams 
using a dynamic slicing technique. Some work on model-based regression 
testing with existing test cases has been done. Korel et al.~\cite{korel02} 
use dependence analysis of Extended Finite State Machine (EFSM) models 
to reduce the size of regression test suites. 
Naslavsky et al.~\cite{naslavsky10} present a test selection technique
that selects test cases for the modified models by establish traceability 
relationships among entities in models and existing test cases. 

While these existing techniques and approaches have motivated us, 
our work is different from theirs in the following aspect.
Our work focuses on evolving web applications that
require frequent patches and regression testing. This means that, unlike
other approaches, we focus only on the areas affected by code changes
and generate test cases using program slicing. Some work on regression
testing for web applications~\cite{dobolyi09, elbaum03} has been done,
but their focus is different than ours. Dobolyi and Weimer~\cite{dobolyi09} 
present an approach that automatically compares the outputs from similar 
web applications to reduce the regression testing effort.
Elbaum et al.~\cite{elbaum03} present a web application testing technique 
that utilizes user session data considering regression testing contexts.

Another research area that is relevant to our work is program slicing.
Program slicing is a heavily researched area of software engineering.
Weiser~\cite{weiser81} formally defined program slicing as a way to
determine which areas of a program are affected by which variables
in a specific line of source code. Agrawal et al.~\cite{agrawal90}
expanded this work by determining the statements which can be affected
by a variable occurrence for a specific input value.
Horwitz et al.~\cite{horwitz88, horwitz90} also contributed by developing a way
to generate slices that traversed multiple procedures using system
dependence graphs. Gupta et al.~\cite{gupta92nov, gupta96jun} have
applied program slicing techniques to the area of selective regression
testing. Ricca and Tonella~\cite{ricca02} presented an approach to construct
program dependency graphs for web applications and addressed problems with
handling event/hyperlink and dynamic nature of web applications.   

