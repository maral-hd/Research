
\subsection{Impact Analysis}

%To analyze the impact of software modifications on a web
%application, we implemented a static program slicer for web
%applications written in the PHP programming language. 
%The benefit of using a program slicer is that it provides 
%a complete way to reveal which areas of the application will 
%be affected by software modifications~\cite{agrawal90}.
%
%Using the ASTs we constructed during the preprocessing steps, 
%we perform the impact analysis that identifies the areas of 
%the application that have been affected by code changes. 
%For impact analysis, three steps are involved: PDG generation,
%program differencing, and program slicing. 
%
%To build PDGs from ASTs, we follow the same approach
%used by Harrold et al.~\cite{harrold93}. 
%First, the PDG generator constructs control flow graphs. 
%During this phase, the PDG generator builds control flow graphs 
%for all methods and functions that have been declared. 
%To construct an inter-procedural control flow graph, all method 
%or function calls within the control flow graphs are linked to 
%their corresponding control flow graphs.
%Next, the PDG generator analyzes the data by performing a def-use 
%analysis, and it gathers data dependency information for variables. 
%%The PDG generator uses a bit-vector to track the 
%%definitions and uses of variables throughout the application. 
%
%After constructing PDGs for two consecutive versions,
%we need to determine which files have been changed from the
%previous version and which statements within these files have
% been changed. To perform this task, we use a textual differencing 
%approach similar to that of the standard Unix diff tool. In our 
%approach, the differencing tool compares two versions of program 
%dependence graphs, in XML format, that contain the program structure 
%and statement content. The differencing tool analyzes each program 
%and applies a longest common subsequence approach on the PDG 
%representation of the statements.
%After the tool determines which statements have been edited, added, 
%or deleted, it provides this information to the program slicer.
%
%Finally, the program slicer creates program slices by using
%the program modification information. Program slicing is a technique 
%that can be used to trace control and data dependencies with 
%an application~\cite{weiser81}, and in this work, it is used for 
%calculating the code change impact set. Our program slicer performs 
%forward slicing by finding all statements that are data dependent on 
%the variables defined in the modified statements and produces a program 
%slice following these dependencies. It then performs backward slicing 
%on the modified statements to determine the statements upon which each 
%modified statement is data dependant. For each modified statement, 
%the forward and backward slices are combined into a single slice.

