%\vspace*{-3pt}
\section{Threats to Validity}
\label{sec:validity}
%\vspace*{-2pt}

The primary threat to validity of this study is the amount of
user session data and the type of users who participated in this
study. For the proprietary application that we used in this study,
we collected user interaction data for a long period time; 
the collected data was created by actual users of 
the application. However, for the two open source applications, 
the period of time over which we collected user interactions was relatively 
short, and the participants were not domain experts or regular users 
of the applications, so their usage patterns had wide variations.
This threat can be addressed by performing additional studies that
monitor user interactions over a longer time period among a wider population,
and by considering industrial applications and different types of 
applications (e.g., mobile applications).
 
%Another threat to validity is the choice of algorithms that classify 
%components' frequency access ranking and analyze change impact. 
%In this study, we applied various algorithms to create our classification
%model, but many other classification algorithms (e.g., decision tree and
%apriori algorithms) are available, and they could produce different results.
%The results can vary depending on the type of classification algorithms, 
%the parameters set for classification algorithms, the variables being analyzed, 
%and the environmental settings. 

Another concern involves the bug reports that we used.
Our classification prediction values for designing linear models 
in the change history analysis were generated from bug history that 
was reported by actual users.
Further, using these bug reports, we measured the coefficient of other 
variables to create our linear model for change history analysis.
Because our bug report data is not comprehensive and contains 
only those bugs accrued 
up to the time that we stopped collecting data, 
and because there might 
be other bugs that have not been reported yet or that might occur 
later, there is a possibility that the bug reports are biased.

