
\subsection{Preprocessing}
\label{sec:preprocess}

%Before we performed the impact analysis on two consecutive 
%versions of PHP web applications, we performed 
%two preprocessing steps to address the challenges associated 
%with analyzing a high-level, dynamic programming language. 
%The following subsections describe each step in detail.
%
%\subsubsection{File Preparation}
%\label{sec:fileprep}
%
%PARTE uses abstract syntax trees (ASTs) to generate PDGs.
%To create ASTs from PHP programs, we used PHC (a PHP compiler),
%which is a publicly available, open source tool.
%We used PHC because it generates the ASTs that contain detailed 
%information about variable data types that is necessary for input 
%data constraint gathering and resolution. 
%
%However, there are a couple problems with PHC. 
%PHC cannot properly parse \textit{include} and \textit{require} 
%functions that contain variables instead of constant strings. 
%Futher, PHC cannot properly evaluate the expressions if any 
%concatenation or defined values are used in the \textit{include} 
%or \textit{require} function.
%Using defined values and runtime concatenation 
%to include a specified file in a location defined during the 
%installation of the web application is a common practice.
%
%Since PHC cannot resolve non-trivial \textit{include} and 
%\textit{require} function calls, we built a parsing tool that searches 
%for them through the web application source code and then locates, 
%defines, and stores them in a dictionary. When the parsing tool locates 
%\textit{include} or \textit{require} function names, it resolves 
%them using the dictionary that maintains defined values. 
%If \textit{include} or \textit{require} functions are completely 
%resolved, the tool inserts their code into the associated 
%source code. This process is repeated until the tool 
%visits every \textit{include} and \textit{require} function.
%
%Another common feature of many dynamic web applications is to specify 
%which files to include at runtime. One typical example of this feature 
%is that a user selects a language preference when the web application 
%is presented. For cases like this one, we manually assign a 
%constant value to the environmental variable in the application. 
%
%\subsubsection{AST2HIR}
%\label{sec:ast2hir}
%
%After we obtained the preprocessed PHP files from the previous step,    
%we generate ASTs for these PHP files using PHC.
%Then, we covert AST to a high-level intermediate representation (HIR) 
%to simplify PDG generation. PHC already provides a function that 
%generates HIR code from an AST, but the HIR code generated by PHC does 
%not preserve consistent variable names across versions of the same 
%application, so it is not suitable for our regression testing tasks. 
%PHC's HIR is focused on code optimization for performance efficiency.
%
%PHC uses a typical three-address code generation approach and creates 
%variable names by incrementing a numerical suffix on a fixed prefix. 
%Thus, if a single statement in a program is added, deleted, 
%or modified, every statement that comes after it in the internal 
%representation could have variable names that do not correspond with 
%the variable names in the original version's internal representation. 
%Our regression testing framework requires consistent variable names 
%to properly generate an impact set given two different versions of 
%the PHP program.
%To resolve this issue, we implemented an AST2HIR tool that maintains 
%consistent variable names while generating a three-address 
%code representation of the program. Instead of simply incrementing 
%based on the order variables appeared in a source file, we used 
%a hashing function on the values from the control dependency 
%information combined with the expression values within a statement 
%from the AST to determine variable names.
%
%There are two main types of symbols in the AST: statements and 
%expressions. Statements are the symbols that define classes, interfaces, 
%and methods as well as the control flow of the program. Expressions 
%are the rest of the symbols, some of which include arithmetic operators, 
%conditional expressions, assignments, method invocations, and logical 
%operators. The AST that PHC generates is in XML format, and the AST2HIR 
%converter parses the XML tags in a depth-first manner. 
%%There is a difference between the parsing method of statements and 
%%expressions. Statements are parsed before their child nodes are parsed 
%%while expressions are only parsed after their child nodes.
%When a complex statement (``complex'' being defined as an expression 
%that is not in formatted, three-address code) is encountered while 
%parsing, it is split into multiple expressions such that all parts 
%are not complex. These parts are given a unique variable name, 
%and assignment expressions are generated to assign the value of 
%the non-complex parts to their corresponding unique variables.
%Some statements are also transformed to work in the PDG generator. 
%For instance, a ``for'' loop is simplified into a ``while'' loop. 
%``Switch'' statements are converted into equivalent code using a 
%``while'' loop, ``if'' statements, and ``break'' statements.
%
%The AST2HIR tool then produces a PHP program that is functionally 
%equivalent to the original program but in the high-level internal 
%representation. An AST for this HIR program is then created using 
%PHC. This HIR version of the AST is used to generate the PDGs. 
%Throughout the paper, we refer to this HIR version of the AST
%simply as AST.
%
