\vspace*{2pt}
\section{Related Work}
\label{sec:related-work}

This section discusses two topics related to our research:
test case prioritization techniques and recommender systems.

\subsubsection*{Test Case Prioritization}
%Test case prioritization techniques reorder test cases to maximize
%some objective functions, such as detecting defects as early as possible.
Due to the appealing benefits of test case prioritization in practice, 
many researchers have proposed various  
techniques.These techniques help engineers discover faults
early in testing, which allows them to begin debugging earlier.
%In this case, entire test suites may still be executed, which avoids 
%the potential drawbacks associated with omitting test cases. 
Recent surveys ~\cite{catal13, marksurvey} provide a comprehensive 
understanding of overall trends of the techniques and suggest areas for improvement.
Depending on the types of information available, various test case
prioritization techniques can be utilized, but 
the majority of prioritization techniques have used source code
information to implement the techniques.
For instance, many researchers have utilized code coverage information
to implement prioritization techniques~\cite{elbaum02feb, kim02may,rothermel01oct}. 
Although this approach is na\"{\i}ve, many empirical
studies have shown that this approach can be effective~\cite{cost3, 
cost1, Malishevsky02, myra}.
%Recent prioritization techniques have used other
%types of code information, such as slices~\cite{jeffrey06sep}, change
%history~\cite{sherriff07}, code modification information, and fault
%proneness of code~\cite{mirarab07}.

Further, some researchers have used software risk information in testing approaches.  
For instance, Frankl and Weyuker ~\cite{weyuker} introduced two risk related measures of 
software testing effectiveness, which are expected detected risk and expected risk reduction
and investigated the effectiveness of these two measures on testing techniques. 
%Chen et al.~\cite{chenRisk} presented a black box
%regression test selection technique, which is a customer-oriented
%and risk-based.
Hettiarachchi et al.~\cite{risk} proposed a new test case
prioritization technique. Their technique uses risk levels of potential defects
to detect risky requirements then it prioritizes test cases by mapping the related 
test cases and these requirements.

More recently, several prioritization techniques 
utilizing other types of information have also been proposed. 
For example, Anderson et al. applied telemetry data to compute fingerprints 
to extract usage patterns and for test prioritization~\cite{jeff16}.
Memon and Amalfitano performed a study applying telemetry 
data to generate usage pattern profiles~\cite{memongui}.  
%In another study, Amalfitano et al. built finite state models
%based on usage data that they collected from rich Internet applications ~\cite{rich}. 
%Carlson et al.~\cite{ryan} presented clustering-based techniques that
%utilize real fault history information including code coverage.
%Anderson et al.~\cite{jeff14} investigated the use of various code features
%mined from a large software repository to improve regression testing techniques.
Gethers et al. presented a method  that uses textual change of source code
to estimate an impact set ~\cite{kagdichange}. 

\subsubsection*{Recommender Systems}  
Recommender systems are software engineering tools that make 
the decision making process easier by providing a list of relevant items.
%There are three primary  categories in recommender systems:
%content-based algorithms, collaborative filtering algorithms, 
%and hybrid approaches ~\cite{recomsurvey05}.  
%Recommender systems are commonly used by users in their daily routines, 
%helping in such tasks as finding 
%their target items more easily.
Some widely-used applications that provide recommender systems
are Amazon, Facebook, and Netflix. These applications provide suggestions
to target users based on the user or on item characteristic similarities.  

%Further, in the area of software engineering, due to the decline in hardware 
%facility prices, a variety of information is collected by software 
%providers, such as change history, issue reports and databases, user log files, 
%and so on.
With the fast growth of such information, machine learning technologies  
motivate software engineers to apply recommendation systems in software 
development. Recommender systems in software engineering have been applied  
to improve software quality and to address the challenges of the development process~\cite{rssebook}.  
For instance, Murakami et al. ~\cite{murakami} proposed a technique that 
uses user editing activities  detecting code relevant to existing methods. 
%Christidis et al. ~\cite{costas} implemented a recommender system 
%to display developer activities by using information artifacts with quantitative metrics. 
Danylenko and Lowe provided a context-aware recommender system 
to automate a decision-making process for determining the efficiency of 
non-functional requirements ~\cite{contextawar}.

As discussed briefly earlier, many types of information are available 
for implementing test case prioritization techniques.
In this research, we collected over 2,000 user sessions from 
three different web applications and gathered the change history of each application. 
Our research seeks to apply item-based collaborative filtering algorithms 
to generate a recommendation list for test prioritization.
To our knowledge, our recommender system-based prioritization technique is novel 
and has not yet been explored in regression testing.

