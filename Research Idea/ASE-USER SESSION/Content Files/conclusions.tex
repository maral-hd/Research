\vspace*{3pt}
\section{Conclusions and Future Work}
\label{sec:conclusions}

In this research, we proposed a new recommender system to 
improve the effectiveness of test case prioritization. 
Our recommender system uses three datasets (code coverage, change history, and user sessions)
to produce a list of the riskiest components of a system for regression testing.
We applied our recommender system using two open source applications and 
one industrial application to investigate whether our approach 
can be effective compared to five different control techniques.
The results of our study indicate that our recommender system can  
improve test case prioritization; also, the performance of our approach 
was particularly noteworthy  when we had a limited time budget. 

Because our initial attempt to use recommender systems in the area of regression 
testing showed promising results, we plan to investigate this approach further
by considering various algorithms (e.g., a context-aware algorithm and a hybrid algorithm), 
the characteristics of applications, and the testing context.
%For example, as we discussed earlier, there are some limitations in collaborative 
%filtering.
In future research, we plan to investigate other approaches 
to address a sparsity problem of our recommender system by applying an associative
retrieval framework and related spreading activation algorithms
to track user transitive interactions through their previous interactions. 
To address the ``New Item Problem,'' we plan to apply 
knowledge-based techniques such as case-based reasoning. 
%Further, in this study, we did not evaluate our approach considering
%cost-benefit tradeoffs, so we would like to investigate this aspect in a future study. 
Also, we wish to apply our recommender system to different software domains 
such as mobile applications.
and perform additional studies that
monitor user interactions over a longer period of time.
%Also, we wish to apply our recommender system to different software domains 
%such as mobile applications  and perform additional studies that
%monitor user interactions over a longer period of time.

