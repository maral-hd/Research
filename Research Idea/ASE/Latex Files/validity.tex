\section{Threats to Validity}
\label{sec:validity}

The primary threat to validity of this study is the amount of
user session data and the type of users who participated in this
study. For the private application that we used in this study,
we collected user interaction data for a long period time; 
the collected data was created by actual users of 
the application. However, for the two open source applications, 
the period of time that we collected user interactions was relatively 
short, and the participants were not domain experts or regular users 
of the applications so their usage patterns had wide variations.
This threat can be addressed by performing additional studies that
monitor user interactions over a longer time period among a wider population,
by considering industrial applications and different types of 
applications (e.g., mobile applications).
 
Another threat to validity is the choice of algorithms that classify 
components' frequency access ranking and analyze change impact. 
In this study, we applied various algorithms to create our classification
model, but many other classification algorithms (e.g., decision tree and
apriori algorithms) are available, and they could produce different results.
The results can vary depending on the type of classification algorithms, 
the parameters set for classification algorithms, the variables being analyzed, 
and the environmental settings. 

There is another concern regarding the bug reports that we used.
Our classification prediction values for designing linear models 
in the change impact analysis were generated from bug history that 
was reported by actual users.
Further, using these bug reports, we measured the coefficient of other 
variables to create our linear model for change impact analysis.
Because our bug report data is not comprehensive and contains 
only those bugs accrued 
until the time that we stopped collecting data, 
and because there might 
be other bugs that have not been reported yet or that might occur 
later, there is a possibility that the bug reports are biased.


%In this section, we describe the internal and external
%threats to the validity of our study. We also describe 
%the approaches we used to limit the effects of these threats.
%
%{\em Internal Validity:}
%The outcome of our study could be affected by the choice of 
%program analysis. We applied static analysis on a dynamic programming 
%language, PHP. To do so, we had to change many statements in the 
%program during the file preparation phase, removing and fixing 
%many dynamic environment variables. However, we carefully examined 
%our process to minimize the adversary effect that might be introduced
%into the converted files.  
%
%{\em External Validity.}
%We used open source web applications for our study, so these programs 
%are not representative of the applications used in practice, and thus
%we cannot generalize our results. However, we tried to reduce this 
%threat by using five non-trivial sized web applications with multiple 
%versions that have been utilized by many users. 

