
\subsection{Data and Analysis}
\label{sec:data1}

%In this section, we present the results of this study considering
%each research question. 
%
%\subsection{RQ1 Analysis}
%
%RQ1 investigates whether the use of the recommender system can help
%improve the effectiveness of test case prioritization techniques.
%Figure~\ref{fig:All} shows the results combined for all applications.
%The horizontal axis shows APFD scores, and the vertical axis shows 
%prioritization techniques:
%``$T_{ch}$'' (change impact-based), ``$T_{mfw}$'' (frequent web form-based), 
%``$T_{mfc}$'' (frequent component-based), ``$T_{r}$'' (random order) and 
%``$T_{hcf}$'' (recommender system-based). 
%
%As shown in Figure~\ref{fig:All}, the proposed technique ($T_{hcf}$) 
%yielded higher APFD scores than all control techniques while its data 
%distribution is slightly wider than others. 
%In particular, $T_{hcf}$ outperformed $T_{r}$ by 56\% in APFD 
%score (the median APFD values for $T_{hcf}$ and $T_{r}$ are 80.59 and
%59.39, respectively). 
%
%The first three control techniques produced very similar results 
%with marginal differences.
%To understand how these techniques performed for each individual
%application, we constructed boxplots for each application.
%
%
%
%Overall, the experimental results showed that our proposed recommender 
%system-based technique performed better than all the control techniques 
%across all three applications. 
%Among the control techniques, the random performed performed worst across all applications,
%whereas the change history-based technique produced slightly better and more stable   
%results for all applications.
%
%\begin{table*}[!ht]
%	\caption{NAPFD Scores on Average.}
%	\vspace*{-10pt}
%	\begin{center}
%		\begin{tabular}{|c|c|c|c|c|c|c||c|c|c|c|}\hline
%			Application & Test Exe. & \multicolumn{5}{c||} {Techniques} 
%			& \multicolumn{4}{c|} {Improvement Rate over Control} \\\hline \hline
%			& Rate (\%)  & $T_{ch}$ & $T_{mfw}$ & $T_{mfc}$ & $T_{r}$ & $T_{hcf}$ 
%			& $T_{hcf}/T_{ch} $& $T_{hcf}/T_{mfw} $ & $T_{hcf}/T_{mfc}$ & $T_{hcf} /T_{r} $  \\\hline \hline
%			
%			&10	&18.54	&12.17	&15.12	&14.33	&23.37	&26\%	&92\%	&54\%	&63\%	\\
%			&20	&21.31	&12.88	&16.78	&15.51	&29.98	&40\%	&132\%	&78\%	&93\%	\\
%			&30	&28.86	&21.19	&25.12	&18.47	&40.95	&41\%	&93\%	&63\%	&121\%	\\
%			&40	&40.42	&29.86	&38.68	&25.59	&55.3	&36\%	&85\%	&42\%	&116\%	\\
%			DASCP &50	&54.34	&36.21	&42.9	&39.15	&67.07	&23\%	&85\%	&56\%	&71\%	\\
%			&60	&61.7	&47.31	&50.34	&45.29	&70.42	&14\%	&48\%	&39\%	&55\%	\\
%			&70	&68.34	&56.96	&65.97	&60.74	&76.67	&12\%	&34\%	&16\%	&26\%	\\
%			&80	&75.41	&69.04	&71.14	&64.88	&84.01	&11\%	&21\%	&18\%	&29\%	\\
%			&90	&83.35	&77.28	&77.69	&67.11	&89.83	&7\%	&16\%	&15\%	&33\%	\\
%			&100	&90.16	&84.21	&79.22	&70.91	&94.14	&4\%	&11\%	&18\%	&32\%	\\\hline \hline
%			
%			&10	&17.54	&12.17	&15.75	&8.33	&28.14	&60\%	&131\%	&78\%	&237\%	\\
%			&20	&29.28	&19.88	&28.35	&12.51	&47.9	&63\%	&140\%	&68\%	&282\%	\\
%			&30	&38.86	&27.19	&35.45	&15.57	&55.28	&42\%	&103\%	&55\%	&255\%	\\
%			&40	&44.42	&34.86	&48.68	&27.59	&65.06	&46\%	&86\%	&33\%	&135\%	\\
%			nopCommerce&50	&58.34	&39.16	&54.94	&35.91	&74.78	&28\%	&90\%	&36\%	&108\%	\\
%			&60	&62.7	&51.42	&57.08	&41.45	&81.49	&29\%	&58\%   &42\%	&96\%	\\
%			&70	&68.14	&55.03	&59.97	&50.02	&86.38	&26\%	&56\%   &44\%	&72\%	\\
%			&80	&76.22	&60.21	&64.14	&58.15	&89.87	&17\%	&49\%	&40\%	&54\%	\\
%			&90	&79.4	&66.01	&70.22	&60.17	&95.14	&19\%	&44\%	&35\%	&58\%	\\
%			&100	&82.16	&68.32	&76.04	&63.91	&97.06	&18\%	&42\%	&27\%	&51\%	\\\hline \hline
%			
%			&10	&26.54	&23.17	&29.12	&20.33	&41.02	&54\%	&77\%	&40\%&	101\%	\\
%			&20	&44.28	&42.88	&45.12	&24.51	&66.98	&51\%	&56\%	&48\%	&173\%\\
%			&30	&60.86	&44.19	&50.12	&28.57	&73.28	&20\%	&65\%	&46\%	&156\%	\\
%			&40	&63.42	&51.51	&53.68	&31.59	&75.06	&18\%	&45\%	&39\%	&137\%	\\
%			Coevery &50	&64.34	&55.74	&58.3	&39.62	&77.1	&19\%	&38\% 	&32\%&	94\%	\\
%			&60	&66.7	&56.33	&60.41	&44.09	&78.65	&17\%	&39\% 	&30\%	&78\%\\
%			&70	&68.19	&59.86	&62.97	&46.11	&81.13	&18\%	&35\%	&28\%	&75\%	\\
%			&80	&70.67	&66.1	&65.14	&57.91	&83.2	&17\%	&25\%	&27\%	&43\%	\\
%			&90	&71.81	&73.14	&66.03	&59.01	&86.69	&20\%	&18\%	&31\%	&46\%	\\
%			&100	&72.16	&76.21	&68.22	&62.91	&87.23	&20\%	&14\% 	&27\%	&38\%	\\\hline 
%			
%		\end{tabular}
%		\end {center}
%		\label{tab:napfd}
%		\vspace*{-5pt}
%	\end{table*}
%	
%	\subsection{RQ2 Analysis}
%	
%	RQ2 investigates whether the use of the recommender system can improve the effectiveness
%	of test prioritization when we have a limited budget for testing. 
%	In RQ2 analysis, we measured NAPFD, which is a normalized ratio of APFD, when 
%	our resources were not consistent.
%	In this experiment, first we executed 10\% of our test cases, and we continued to execute the
%	test cases in increments of 10\% of the total until they had all been executed. 
%	executed to see whether we could improve the fault detection rate given a 
%	time constraint dictating that running 100\% of the test cases at one time was not feasible. 
%	
%	Table~\ref{tab:napfd} shows the results of our three applications.
%	This table shows the results of two primary analyses. The first part of the table 
%	presents the NAPFD scores on average, and the second part shows the 
%	improvement rates of the heuristic technique over the four other control techniques. 
%	
%	By examining the numbers in the table, we can observe that the improvement
%	rates of our heuristic technique over the control techniques vary widely. 
%	When we compared the heuristic with $T_{ch}$, the improvement rates ranged from
%	4\% to 41\% for DASCP, from 17\% to 63\% for nopCommerce, and from 17\% to 54\%
%	for Coevery.
%	When compared with $T_{mfw}$, the improvement rates ranged from
%	11\% to 132\% for DASCP, from 42\% to 140\% for nopCommerce, and from 14\% to 77\%
%	for Coevery, indicating significant improvements. 
%	When compared with $T_{mfc}$, the improvement rates ranged from
%	15\% to 78\% for DASCP, from 27\% to 78\% for nopCommerce, and from 27\% to 48\%
%	for Coevery, indicating results similar to those for as $T_{ch}$.
%	As for the comparison with $T_{r}$, the results were more remarkable.
%	The rates ranged from 32\% to 282\% for all three applications. 
%	
%	One outstanding trend we observed in the table is that the improvement 
%	rates are much higher when the time budget is smaller.
%	For example, in the comparison with $T_{ch}$ for nopCommerce, 
%	when 10\% of budget was assigned, the improvement rate was 60\%,
%	but when we had a full budget, the rate dropped to 18\%.
%	A similar trend can be observed across all control techniques and applications.  
%	This indicates that our approach can be more helpful when companies are operating
%	under a tight budget.
%
%	
%	We can infer that a great number of defects can be 
%	detected at a very early stage of test case execution when applying our approach
%	to nopCommerce and Coevery, looking at the figures for those two applications.
%	
	
	


%\begin{table}[!ht]
%	\caption{Test Case Prioritization Techniques}
%	\vspace*{-10pt}
%	\begin{center}
%		\begin{tabular}{ |l|l|l| }
%			\hline
%			%\multicolumn{3}{ |c| }{ } \\\hline
%			Group & Technique & Description \\ \hline
%			\multirow{4}{*}{Control} 
%			& xx & Lines of code. \\
%			& xxx Coupling &  The class coupling.\\
%			& xx of Parameters & The number of parameter \\
%			& xx of Operator &   The number of operator contained in a class.\\\hline
%			\multirow{3}{*}{Heuristic} 
%			& xx & Time when the change was made. \\
%			& xx & The number of revision of a file. \\		
%			& xx & The total number of lines added \\ \hline
%		\end{tabular}
%		\end {center}
%		\label{tab:techniques}
%		\vspace*{-5pt}
%	\end{table}






