\section{Introduction}
\label{sec:introduction}

Software systems undergo many changes during their lifetime, 
and often such changes can adversely affect the software. 
To avoid undesirable changes or unexpected bugs, 
software engineers need to test the overall 
functionality of the system before deploying 
a new release of the product. One of the common ways 
to evaluate system quality in a sequence of releases is 
regression testing. In regression testing, software engineers 
validate the software system to ensure that new changes 
have not introduced new faults or they don’t affect 
the other parts of the system. However, modern software systems 
evolve frequently, and their size and complexity grow quickly, 
and thus the cost of regression testing can become too expensive [4]. 
To reduce the regression testing cost, many regression testing and 
maintenance approaches including test selection and test prioritization  
have been proposed [34].

To date, the majority of regression testing techniques have 
utilized various software metrics that are available from software repositories, 
such as the size and complexity of the application, code coverage, 
fault history information, and dependency relations among components. 
Further, various empirical studies have shown that the use of a 
certain metric or a combination of multiple metrics can improve 
the effectiveness of regression testing techniques better than other metrics. 
For example, Anderson et al. [5] empirically investigated the 
use of various code features mined from a large software repository, 
showing that these code features can help improve regression testing processes. 
However, we believe that, rather than simply picking one metric over another, 
adopting a recommender system that identifies more relevant metrics by 
considering software characteristics and the software testing environment would provide a better solution.

Recommender systems have been utilized to help reduce the 
decision making effort by providing a list of relevant 
items to users based on a user’s preferences or item attributes. 
For example, companies that produce daily-life applications, 
such as Netflix, Amazon, and many social networking applications 
[17] are adopting recommender systems to provide more personalized 
services so that they can attract more users. Recently, 
recommender systems have been used in software engineering 
areas to improve various software engineering tasks. For example, 
Mens et al. provide a source code recommendation system to help 
the developer by giving hints and suggestions or by correcting an existing code [33]. 
%Zanjani et al. performed a study by developing a 
%recommender system for code review based on historical 
%contributions of prior reviews [35]. Anvik et al. conducted 
%research that applied machine learning techniques to developers 
%and bug history to make suggestions about “who should fix this bug?” [6]. 
%While many software engineering techniques have started to incorporate 
%recommendation systems, no researchers have investigated 
%the use of recommender systems in the area of regression testing.

According to previous studies of recommender systems, 
a majority of proposed approaches focus on recommending the most 
relevant items without considering contextual information. 
However, contextual information such as time and cost can improve 
the efficiency of the recommended list by providing the most 
relevant items per their cost ratio. For example, in a 
typical test prioritization scenario, testers assume that 
all test cases and faults have same cost while, in practice, 
there is a wide range of test costs and fault impacts on the 
system that testers do not put into a context when 
providing a list of most important test cases. Moreover, 
there are variety of goals for test prioritization. 
For example, a tester's goal could be: increasing the code coverage, 
increasing the fault detection rate in a shorter 
time and increasing the system reliability. 
Thus, for a given prioritization goal, 
different prioritization techniques might be applicable. 
In any case, the intent behind the choice of a prioritization technique 
is to increase the likelihood that the prioritized test suite can 
better meet the goal than would a random order of test cases.

However, there are several research have been done 
in context-aware regression testing area but they have not been considered the other
aspect of the regression causes while applying context information. 
For example, Do etl, [] provided improved cost-benefit models for use 
in assessing regression testing methodologies,
that incorporate context and lifetime factors. % paraph beshe
In their study they provide a cost model based on contextual information without 
considering the other factors that can be the causes of regression issues such as change history. 
Zhange etl, [] proposed a time-aware test case prioritization 
using integer linear programming, but their work is limited to the code coverage
information rather than test cost. Qu etl, [] used combinatorial interaction
testing techniques to model configuration samples for use in regression testing 
and they proposed a normalized fault detection rate model, still in this
study they did not consider the testing cost as a cost factor.   

Therefore, in this paper we present a 
recommender system that uses test execution time as test cost,  
code dependency and change history as a metrics for measuring 
the criticality portion of code that accounts for varying 
test case and fault costs. To further investigate these metrics 
and its applications, we present prioritization techniques that 
account for these varying costs, and results of a case study in 
which we apply these techniques under several different test cases 
and fault severity cost distributions.
 
%Therefore, in this paper we present a new context aware 
%recommender system that uses test execution time and 
%fault severity as the contextual metrics of test cases 
%for measuring the rate of fault detection that accounts for varying 
%test case and fault costs. To further investigate these metrics 
%and its applications, we present prioritization techniques that 
%account for these varying costs, and results of a case study in 
%which we apply these techniques under several different test cases 
%and fault severity cost distributions. 

In order to implement our recommender system, 
we used relevant information of 
test cases and the applications under test and we showed that using 
this information can improve the effectiveness of test prioritization 
technique when we consider test cost. We implemented a multi-objective test case 
prioritization technique by applying our recommender system and 
performed an empirical study using two open source software systems 
and one commercial system. The results of our study show that our 
proposed approach can improve the effectiveness of test case prioritization 
compared to four other control techniques. The main contributions of 
this research are as follows: (1) We propose a cost-effective 
recommender system that improves test prioritization and 
(2) we perform empirical evaluations of the proposed technique 
and four other control techniques.

The rest of the paper is organized as follows. In Section 2, 
we discuss the approach used in this research and formally 
define context aware recommender systems. 
Sections 3 and 4 present our empirical study, including the design, 
results, and analysis. Section 5 discusses the results and the 
implications of these results. Section 6 discusses potential 
threats to validity and Section 7 presents background and related work. 
Finally, in Section 8, we provide conclusions and discuss future work.






