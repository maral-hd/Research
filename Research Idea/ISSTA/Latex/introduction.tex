\section{Introduction}
\label{sec:introduction}
% what is the problem?
During application's life it may change several times, and one change can affect the entire system.
To avoid the undesirable change or bugs, system testers need to  
test the overall functionality of the system before deploy the new release of the system.
One of the most common way to evaluate the system quality in sequence of release is regression testing.
In regression testing, testers check the software to ensure that new changes have not 
introduced new faults or it did not effect the other parts of system.
Applying regression testing before deploying new release of the software, make the 
change process safer and more confident for developers.

As software grow, the size of test suite grow in a same
manner, eventually testing the entire system can become expensive and may take
40\% of whole project budget[].  Moreover, it is not feasible to test every single 
function of the system especially for large scale systems it may takes several days 
to execute test cases. Several techniques have been proposed to solve this issue 
%by reducing the cost and effort of system testing.
such as, test reduction which is a 
technique that select subset of test cases that can address most of the issues, 
test selection also is a similar to test reduction but the main difference is 
this technique, select test cases based on their correlation with the changes 
in tow versions of the subject, and test case prioritization [].

% what approaches have been proposed?
Test case prioritization is a technique to catch the faults earlier, 
by executing the most important test cases first.
So far, many prioritization strategies have been proposed by researchers [] and
many techniques have been purposed such as : greedy techniques [], change impact analysis [],
clustering approaches[], user session based techniques [], etc. 
but most of the investigated studies are based on two main features: code coverage and 
code complexity metrics. Meanwhile, there are many other features of the software 
that can be a factor of failure or can be a hint to find the failure causes. 
One interesting feature of web application is that recording users interaction data
is a way easier compare to other types of applications. 
Different types of data from users interaction can be
monitored, such as users sessions, cookies, telemetry data, users IP address etc. 
Storing and collection log files and users sessions nowadays are less costly and more 
accessible compare to past decades, due to lower price of the IT services and hardware.
We can store information in database or cloud system 
rather than log files which was a traditional way to storing system back up. 
Having all these types of data and facilities, enables testers to identify most frequently used
components which impact the system strongly. 
% jeff paper

% why they didint work?
Previous studies shown that change history has more significant role in 
terms of finding faults than code metrics []. 
Although, there are several research have been done about the efficiency of 
change metrics for defect prediction, such as [] but, researchers only focused on 
measuring the causes and effect of the changes in software. 
Moreover, users session and telemetry data 
have been used as a factor for research in software testing. 
For example, Elbaum et al.~\cite{elbaum2005profiling} performed a study on regression 
prioritization focusing prioritization at a product level 
by using telemetry from installations. 
Their work was extended in several domain such as, performance issue detection ~\cite{parsons2007automatic},
bug detection ~\cite{wang2008approach}, reliability testing of rule-based 
systems ~\cite{avritzer1996reliability} and failure reproduction ~\cite{jin2012bugredux}. 
Sampath et al. ~\cite{Sampath2008} applied multiple techniques for prioritizing test cases 
by using user sessions as a factor for their techniques. 


Lets conciser a typical test prioritization based on change history. 
In this scenario, testing is based on executing any component that has been 
changed in new release of software [1-2], % ref: A Comparison of Coverage-Based and Distribution-Based Techniques for Filtering and Prioritizing Test Cases
while this could be a minor change that 
does not have a considerable impact on the entire system, 
such as renaming a function or can be a 
component that is used seldom by users, etc. 
In another scenario, we reorder test cases based on the component frequency, 
meanwhile, there might be no change in this particular component compare to previous version.
For example, generally, core components in a system are mostly used components in the entire system 
while, those components due to their significant role have been tested several times and rarely change. 
Due to above challenges, we belive that current studies in test prioritization 
is neither efficient nor sufficient to address the test priritization problem. 
%We believe that applying multiple factors
%rather than focusing on a specific viewpoint would help improve test case prioritization techniques.

%such as, extracting usage patterns from telemetry data besides analysis of change history. 


% what is your solution and why this work?
In this paper we proposed an item based collaborative filtering recommender system 
for test case prioritization in web application which, uses the telemetry 
data as input for users rating and change history of applications 
as items information. The output of our recommender system is the prioritized test 
cases in a way to obtain the better results in terms of finding faults earlier. 
We applied our recommender system in three open source system and one commercial system.
Our hypothesis is: Combination of various metrics can lead us to design 
more efficient model for fault detection. Our approach focuses on both users 
interaction and change history metrics to detect most risky components, then by 
automating the detection of regression issues, we are reducing the analysis effort for 
test prioritization. Our recommender system, may not locate the regression issue sources 
directly, but it will provide a  list of top potential risky components. 

The main contributions of this study are as follows:
1) Proposing a new hybrid test prioritization technique for web applications domain, 
2) Empirical evaluation of proposed technique and three other control techniques, and
3) Highlighting issues occurred during applying the proposed technique and its limitation 
and guidance to testers upon the results of our study. 

The rest of the paper is organized as follows. In Section 2, we
discuss the approach used in the research as well as formally define
collaborative filtering recommender systems. In Section 3, we detail the empirical study
that we performed. Section 4 provides the results of the study. 
Section 5 discusses the results and the implications of these results, and
Section 7 presents related work. Finally, in Section 8, we provide
conclusions and discuss future work.




%The usage of web application in nowadays is very clear for verity of people 
%from IT expert to someone with minimum knowledge of the Internet. 
%Not only many business are dependent on web applications but also
%web application grow very fast, and number of users of web application are
%growing exponentially, these factor made software engineer to spend 
%more time on deign robust system with minimum failure risk. 
%One small bug can cost so much money and time for the company.
%So far many research have been done to eliminate the risk of
%web applications failure such as test driven development, regression testing, 
%and many more but they are not sufficient neither optimum to detect the faults 
%when it occurs. 
%
%During application life it may change several times, and one change can affect the entire system.
%To avoid the undesirable change or bugs, system testers need to  
%test the overall functionality of the system before deploy the new release of the system.
%But still there might be some issues that can be observe later after deployment.
%One of the most common way to evaluate the system quality is regression testing.
%In regression testing, tester are seeking to find the cause of system failure 
%between old and new versions. Advantage of regression testing is, it make the 
%change process safer and more confident for developers.
%By fast increasing size of software systems, 
%size of test suits grow exponentially as a result, testing the whole system is a time consuming process 
%and can take several days, also it is not feasible to test every function of the system.
%Test case prioritization is 
%a technique to catch the faults earlier, by executing the most important test cases first.
%There are different ways to reorder test cases such as code coverage based, 
%based on frequency of use, based on system architecture and many other statistical technique [xx].
%
%
%Another feature of web applications which made it more popular
%compare to other types of software systems is users monitoring and recording 
%users activity is a way easier and accessible [bryce and others]. 
%% example of ppl who used telemetry
%% and converting usage
%There are different types of data from users interaction can be
%monitored, such as users sessions, cookies, telemetry data, users IP address etc. 
%having all these information has several benefits but to mention the most significant in testing area is that it gives an opportunity to the system tester
%to extract the pattern of most frequently used components of the system so, they can spend more effort 
%of fixing issues of such a components than others.
%
%In this paper we proposed a collaborative filtering recommender system 
%for test case prioritization in web application which, uses the telemetry 
%data as input data of users interaction and change history of applications 
%as items information. The output of our recommender system is the prioritized test 
%cases in a way to obtain the better results in terms of finding faults earlier. 
%We apply our recommender system in three open source system and one commercial system.
%
%The rest of the paper is organized as follows. In Section 2, we
%discuss the approach used in the research as well as formally define
%telemetry fingerprinting. In Section 3, we detail the empirical study
%that we performed. Section 4 provides the results of the study. 
%Section 5 discusses the results and the implications of these results, and
%Section 7 presents related work. Finally, in Section 8, we provide
%conclusions and discuss future work.
%















