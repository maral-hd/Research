\begin{abstract}

%
%Performance regression is a consequence of software evolution. 
%Code quality and change history play a significant role in detecting performance regression.
%Small changes can significantly degrade software performance. 
%During developing time, some times developers negligent code quality and performance, 
%this carelessness can be a cause of future regression, which will needs 
%more cost and effort to fix in later stages.
%Performance regression testing is an effective way to reveal such issues
%in early stages. 
%Yet because of its high overhead, this activity is usually performed infrequently. 
%Consequently, when performance regression issue is spotted at a certain point, multiple commits might
%have been merged since last testing. Developers have to spend extra
%time and efforts narrowing down which commit caused the problem. 
%Existing efforts try to improve performance regression testing
%efficiency through test case reduction or prioritization.
%In this paper, we propose a new context aware recommender system, which 
%analyses code quality and risk factors and will return a ranking for
%risky components. The analysis consider change history of the system,
%code metrics and number of execution time and interaction level for each component.
%Based on the obtained ranking from our recommender system we will test high risks 
%components first while delaying or skipping testing on low-risk commits.
.
%Companies that provide
%web applications often encounter various security
%attacks and frequent feature update demands from users,
%and when these needs arise, companies need to fix security
%problems or upgrade the application with new features.
%These fixes often involve small patches
%or revisions, but still, testers need to perform regression
%testing on their products to ensure that the changes have not 
%introduced new faults. Performing regression testing on the entire 
%product, however, can be very expensive, and it is not a viable  
%solution for the companies who need a short turnaround time to  
%release patches. One solution is focusing only on the areas of 
%code that have been changed and performing regression testing 
%on them. By doing this, companies can provide quick patches more 
%dependably whenever they encounter security breaches. In this paper, 
%we propose a new regression testing approach that identifies 
%the affected areas by code changes using impact analysis and 
%generates new test cases for the impacted areas by changes using 
%program slices. To facilitate our approach, we implemented 
%a PHP Analysis and Regression Testing Engine (PARTE) and performed 
%a controlled experiment using five open source web applications
%with multiple versions.
%The results show that our approach is effective in reducing 
%the cost of regression testing for a frequently patched web 
%application, and they also expose ways in which that 
%effectiveness can vary with application characteristics
%and versioning frequencies.  

\end{abstract}

